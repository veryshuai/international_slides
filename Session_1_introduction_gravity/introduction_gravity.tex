Basic course structure: - Weeks 1-4, trade - Weeks 5-7, international
finance

\begin{center}\rule{3in}{0.4pt}\end{center}

The textbook: - International Economics: Theory and Policy, Krugman,
Obstfeld, and Melitz, 10th edition. - Two semesters recommended - Online
myeconlab

\begin{center}\rule{3in}{0.4pt}\end{center}

Course evaluation: - 23 October: A four-hour, closed-book written
examination - Based on - Problem sets - Lecture content - Textbook -
Everything from class, homework, or textbook is fair game -
Administrative questions? Tina Østergaard, toe.stu@cbs.dk

\begin{center}\rule{3in}{0.4pt}\end{center}

Problem sets: - Once per week - Some textbook problems, others my own -
I will not typeset solutions

\begin{center}\rule{3in}{0.4pt}\end{center}

Email policy: - Don't send me emails -(unless absolutely necessary) -
Instead\ldots

\begin{center}\rule{3in}{0.4pt}\end{center}

Course wiki: - link at davidjinkins.com - other class-related stuff
there - it's my first time with wiki

\begin{center}\rule{3in}{0.4pt}\end{center}

End of administration section

\begin{center}\rule{3in}{0.4pt}\end{center}

Why should you care about international economics?

\begin{enumerate}
\def\labelenumi{\arabic{enumi}.}
\itemsep1pt\parskip0pt\parsep0pt
\item
  Personally: Denmark is a small, open economy
\item
  Intellectually: Many puzzles remain
\item
  Politically: News articles often incomplete
\end{enumerate}

\begin{center}\rule{3in}{0.4pt}\end{center}
