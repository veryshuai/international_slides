\documentclass{beamer}

\usepackage{graphicx}
\usepackage{hyperref}
\usepackage[latin1]{inputenc}
\usepackage[T1]{fontenc}
\usepackage[english]{babel}
\usepackage{listings}
\usepackage{xcolor,mathrsfs,url}
\usepackage{amssymb}
\usepackage{amsmath}
\usepackage{ifthen}

\usepackage{metricsbeamer} % using the metric beamer style

% The command to define a subsection is '\subsec{}' and NOT '\subsection'.
% This code generates the bar. Don't edit.
\newcommand{\midbarnew}{}
\newcommand{\subsec}[1]
{
  \ifthenelse{\equal{#1}{}}
  {\renewcommand{\midbarnew}{} \subsection{}}
  {\renewcommand{\midbarnew}{ $\mid$ } \subsection{#1}}
}

% change the pictures here, if necessary. logobig and logosmall are the internal names
% for the pictures: do not modify them, just change "hulogo" and "logo". Pictures must be 
% supplied as JPEG, PNG or PDF
%########################################

\pgfdeclareimage[height=2cm]{logobig}{logo} % use hucase instead for the Humboldt-Case Logo
\pgfdeclareimage[height=1cm]{logosmall}{logo}

% use this number to modify the scaling of the headline on titlepage
\def\titlescale{1.0}


\def\authora{Battista Severgnini}	% First Author
\def\affa{Copenhagen Business School} % First Author's Affiliation
\def\authorb{}  % Second Author
\def\affb{} % Second Author's Affiliation
\def\authorc{}  % Third Author
\def\affc{} % Third Author's Affiliation
\def\linka{}	% Link to your institution's/ personal website
\def\linkb{}
\def\linkc{}\def\email{\href{mailto:bs.eco@cbs.dk}{bs.eco@cbs.dk}}	% Your email address

\title[Output and the Exhange Rate in the Short Run]{International Economics - B.Sc. IB\\ 11. Open-Economy Macroeconomics: \\Output and the Exchange Rate in the Short Run
 \\  \footnotesize{May $15^{th}$, 2014}}
\institute{Economics Department CBS}
%Start of the document
\begin{document}

\frame[plain]{% create the titleslide, layout controlled in metricsbeamer
	\titlepage
}

\Section{Plan for Today}

\frame{% how to print
\frametitle{Plan for Today}
Chapter 17:
\begin{itemize}
\item Determinants of aggregate demand in the short run
\item Short run equilibrium for aggregate demand and output (DD curve)
\item Short run equilibrium in the asset markets (AA curve)
\item Short run equilibrium (AA \& DD)
\item Temporary changes in monetary and fiscal policy
\item Permanent changes in monetary and fiscal policy
\item Macroeconomic policies and the current account (XX curve)
\end{itemize}
}


\Section{Chapter 17. Determinants of Aggregate Demand in the Short Run}
\frame{% how to print
\frametitle{}
\begin{center}
\textcolor{blue}{\Huge{\textbf{Chapter 17: Output and the Exchange Rate in the Short Run
}}}
\end{center}
}


\frame{% how to print
\frametitle{Determinants of Aggregate Demand in the Short Run (1)}
Remind:
\begin{enumerate}
\item Long run: flexible prices
\item \textbf{Short run}: prices are sticky (reasons: menu costs, labor contracts, imperfect information...)
\end{enumerate}
In this class: relationship between $E$ and $Y$ ($E=f\left(Y\right)$) in the short run.
}


\frame{% how to print
\frametitle{Determinants of Aggregate Demand in the Short Run (2)}
Aggregate demand $D$ can be expressed by:
\begin{center}
$D=C+I+G+CA$
\end{center}
where
\begin{itemize}
\item $C$: consumption expenditure
\item $I$: investment expenditure
\item $G$: government purchase
\item $CA$: current account
\end{itemize}
}

\frame{% how to print
\frametitle{Determinants of Aggregate Demand in the Short Run: $C$}
\begin{center}
$C=C\left(Y^{d}, R\right)$
\end{center}
Consumption expenditure is a function of 
\begin{itemize}
\item disposal income $Y^{d}=Y-T$ (positively related, but consumption typically increases less than the amount that disposable income increases.)
\item interest rate $R$ (assumption in our model: unimportant)
\item wealth (assumption in our model: non important)
\end{itemize}
}

\frame{% how to print
\frametitle{Determinants of Aggregate Demand in the Short Run: $G, I, T$}
\begin{itemize}
\item Government purchases $G$, the level of taxes $T$, and investment expenditure $I$ are determined by exogenous factors.
\item in reality: $I=I\left(R\right)$
\end{itemize}
}


\frame{% how to print
\frametitle{Determinants of Aggregate Demand in the Short Run: $CA$}
\begin{center}
$CA\approx EXP-IMP$
\end{center}
\begin{center}
$CA=CA\left(\frac{E P^{*}}{P}, Y-T\right)$
\end{center}
Current account is a function of 
\begin{itemize}
\item $\frac{E P^{*}}{P}$: if $\frac{E P^{*}}{P} \Uparrow \Rightarrow EXP \Uparrow, IMP \Downarrow, TOT \Uparrow$
\item assumption: a real depreciation leads to an increase in the current account, i.e. the volume effect dominates the value effect
\end{itemize}
}


\frame{% how to print
\frametitle{Determinants of Aggregate Demand in the Short Run}
The aggregate demand 
\begin{center}
$D=C+I+G+CA$
\end{center}
can be rewritten as 
\begin{center}
$D = C\left(Y �- T\right) + I + G + CA\left(\frac{E P^{*}}{P}, Y �-T\right)$
\end{center}
or 
\begin{center}
$D = D\left(\frac{E P^{*}}{P}, Y �- T, I, G\right)$
\end{center}
}

\frame{
\frametitle{Determinants of Aggregate Demand in the Short Run}
Two effects:
\begin{enumerate}
\item \textbf{Real exchange rate:} $\frac{E P^{*}}{P} \Uparrow \Rightarrow CA \Uparrow \Rightarrow D \Uparrow$
\item \textbf{Disposable income:} $ \left(Y-T\right)\Uparrow \Rightarrow C \Uparrow \Rightarrow CA \Downarrow \Rightarrow D ?$
\end{enumerate}
\begin{itemize}
\item usually consumption expenditure is greater than expenditure on foreign products $\Rightarrow D \Uparrow$
\end{itemize}
}

\Section{Chapter 17. Short Run Equilibrium for Aggregate Demand and Output (DD Curve)}

\frame{
\frametitle{Short Run Equilibrium for Aggregate Demand and Output}
Equilibrium is achieved when the value of income from production (output) Y equals the value of aggregate demand D.
\begin{center}
$Y= D\left(\frac{E P^{*}}{P}, Y �- T, I, G\right)$
\end{center}
}


\frame{
\frametitle{Fig. 17-2: The Determination of Output in the Short Run}
\begin{figure}
	\centering
		\includegraphics[width=0.90\textwidth]{uno.pdf}
	\label{fig:11}
\end{figure}
}


\frame{
\frametitle{DD Schedule (1)}
\begin{itemize}
\item with $\bar{P}$ and $\bar{P^{*}}$ if  $E \Uparrow \Rightarrow CA \Uparrow \Rightarrow D \Uparrow$
\item $\Rightarrow Y \Uparrow$
\end{itemize}
}


\frame[plain]{
\frametitle{Fig. 17-3: Output Effect of a Currency Depreciation with Fixed Output Prices}
\begin{figure}
	\centering
		\includegraphics[width=0.90\textwidth]{due.pdf}
	\label{fig:11}
\end{figure}
}

\frame{
\frametitle{Fig. 17-4: Deriving the DD Schedule}
\begin{figure}
	\centering
		\includegraphics[width=0.90\textwidth]{tre.pdf}
	\label{fig:11}
\end{figure}
}


\frame{
\frametitle{DD Schedule (2)}
DD schedule:
\begin{itemize}
\item combinations of $Y$ and the $E$ at which the output market is in short run equilibrium 
\item slopes upward 
\end{itemize} 
}


\frame{
\frametitle{DD Schedule (3)}
Several changes could cause $DD$ to shift to the right:
\begin{enumerate}
\item $G$: if $G\Uparrow \Rightarrow D \& Y \Uparrow$ 
\item $T$: if $T\Downarrow \Rightarrow C \Uparrow \Rightarrow D \& Y \Uparrow$
\item $I$: if $I\Uparrow \Rightarrow D \& Y \Uparrow$ 
\item $\frac{P}{P^{*}}$: if $\frac{P}{P^{*}} \Downarrow \Rightarrow D \& Y \Uparrow$
\item $C$: if $C\Uparrow \Rightarrow D \& Y \Uparrow$
\item demand of domestic goods with respect to the demand of foreign goods: $D \& Y \Uparrow$
\end{enumerate} 
}


\frame[plain]{
\frametitle{Fig. 17-5: Government Demand and the Position of the DD Schedule}
\begin{figure}
	\centering
		\includegraphics[width=0.90\textwidth]{quattro.pdf}
	\label{fig:11}
\end{figure}
}
\Section{Chapter 17. Short Run Equilibrium in Asset Markets (AA Curve)}


\frame{
\frametitle{Short Run Equilibrium in Asset Markets (1)}
Two sets of assets markets:
\begin{enumerate}
\item Foreign exchange markets 
\begin{center}
$R = R^{*}+ \frac{(E^{e} �- E)}{E}$ 
\end{center}
\item Money market
\begin{center}
$\frac{M^{s}}{P}= L\left(R, Y\right) $
\end{center}
\end{enumerate}
}


\frame[plain]{
\frametitle{Fig. 17-6: Output and the Exchange Rate in Asset Market Equilibrium}
\begin{figure}
	\centering
		\includegraphics[width=0.90\textwidth]{cinque.pdf}
	\label{fig:11}
\end{figure}
}


\frame{
\frametitle{Short Run Equilibrium in Asset Markets: AA Curve}
If $Y \Uparrow$:
\begin{enumerate}
\item $L\left(R, Y\right) \Uparrow$
\item $R \Uparrow$
\item $E \Downarrow$
\end{enumerate}
The inverse relationship between output and exchange rates needed to keep the foreign exchange markets and the money market in equilibrium is summarized as the AA curve.
}

\frame[plain]{
\frametitle{Fig. 17-7: The AA Schedule}
\begin{figure}
	\centering
		\includegraphics[width=0.90\textwidth]{sei.pdf}
	\label{fig:11}
\end{figure}
}

\frame{
\frametitle{AA Schedule}
Several changes could cause $AA$ to shift:
\begin{enumerate}
\item $M^{s}$: if $M^{s}\Uparrow \Rightarrow R \Downarrow \Rightarrow E \Uparrow$: $AA$ shifts up.  
\item $P$: if $P\Uparrow \Rightarrow \frac{M^{s}}{P} \Downarrow \Rightarrow R \Uparrow \Rightarrow E \Downarrow$: $AA$ shifts down.
\item $L\left(R,Y\right)$: if $L\left(R,Y\right)\Downarrow\Rightarrow$ more non-monetary assets $\Rightarrow E \Uparrow$: $AA$ shifts up.
\item $R^{*}$: if $R^{*}\Uparrow E \Uparrow$: $AA$ shifts up. 
\item $E^{e}$: if $E^{e}\Uparrow \Rightarrow D \& Y \Uparrow$
\end{enumerate} 
}

\frame[plain]{
\frametitle{Shifting the AA Curve}
\begin{figure}
	\centering
		\includegraphics[width=0.90\textwidth]{sette.pdf}
	\label{fig:11}
\end{figure}
}

\frame[plain]{
\frametitle{Fig. 17-7: Shifting the AA Curve}
\begin{figure}
	\centering
		\includegraphics[width=0.90\textwidth]{otto.pdf}
	\label{fig:11}
\end{figure}
}
\Section{Chapter 16. Short Run Equilibrium (DD \& AA)}

\frame{
\frametitle{Short Run Equilibrium}
A short run equilibrium means $E$ and $Y$ such that:
\begin{enumerate}
\item equilibrium in the output markets holds (DD): $D=Y$
\item equilibrium in the foreign exchange markets holds (AA): $R = R^{*}+ \frac{(E^{e} �- E)}{E}$ 
\item equilibrium in the money market holds: $M^{s}=M^{d}$
\end{enumerate}
}

\frame{
\frametitle{Fig. 17-8: Short-Run Equilibrium: The Intersection of DD and AA}
\begin{figure}
	\centering
		\includegraphics[width=0.90\textwidth]{nove.pdf}
	\label{fig:11}
\end{figure}
}



\frame[plain]{
\frametitle{Fig. 17-9: How the Economy Reaches Its Short-Run Equilibrium}
\begin{figure}
	\centering
		\includegraphics[width=0.90\textwidth]{dieci.pdf}
	\label{fig:11}
\end{figure}
}



\Section{Chapter 17. Temporary Changes in Monetary and Fiscal Policy}

\frame{
\frametitle{Temporary Changes in Monetary and Fiscal Policy}
\begin{figure}
	\centering
		\includegraphics[width=0.90\textwidth]{nove.pdf}
	\label{fig:11}
\end{figure}
}


\frame{
\frametitle{Temporary Changes in Monetary and Fiscal Policy}
\begin{itemize}
\item \textbf{Monetary policy:} the central bank influences the supply of monetary assets (AA) 
\item \textbf{Fiscal policy:} governments influence the amount of government purchases and taxes (DD)
\end{itemize}
Both policies are expected to be effective in the short run.
}

\frame{
\frametitle{Temporary Monetary Policy}
\begin{itemize}
\item if $M^{s} \Uparrow \Rightarrow R \Downarrow E \Uparrow$
\item $AA$ shifts up
\item Domestic products relative to foreign products are cheaper so that aggregate demand and output increase until a new short run equilibrium is achieved.
\end{itemize}
}


\frame{
\frametitle{Fig. 17-10: Effects of a Temporary Increase in the Money Supply}
\begin{figure}
	\centering
		\includegraphics[width=0.90\textwidth]{undici.pdf}
	\label{fig:11}
\end{figure}
}

\frame{
\frametitle{Temporary Fiscal Policy}
In the short run
\begin{itemize}
\item if $G \Uparrow$ (or $T \Downarrow$) $\Rightarrow D \& Y \Uparrow$
\item $DD$ shifts down
\item $Y \Uparrow \Rightarrow L\left(Y,R\right) \Uparrow \Rightarrow R \Uparrow$
\item $E \Downarrow$
\end{itemize}
}



\frame{
\frametitle{Fig. 17-11: Effects of a Temporary Fiscal Expansion}
\begin{figure}
	\centering
		\includegraphics[width=0.90\textwidth]{dodici.pdf}
	\label{fig:11}
\end{figure}
}



\frame{
\frametitle{Policies to Maintain Full Employment}
Processes can be
\begin{itemize}
\item underemployed
\item natural/potential level
\item overemployed
\end{itemize}
}

\frame[plain]{
\frametitle{Fig. 17-12: Maintaining Full Employment After a Temporary Fall in World Demand for Domestic Products}
\begin{figure}
	\centering
		\includegraphics[width=0.90\textwidth]{tredici.pdf}
	\label{fig:11}
\end{figure}
}

\frame[plain]{
\frametitle{Fig. 17-13: Policies to Maintain Full Employment After a Money Demand Increase}
\begin{figure}
	\centering
		\includegraphics[width=0.90\textwidth]{quindici.pdf}
	\label{fig:11}
\end{figure}
}

\frame{
\frametitle{Policies to Maintain Full Employment}
Policies to maintain full employment are difficult to implement:
\begin{enumerate}
\item people may anticipate the effects of policy changes and modify their behavior (e.g., Barro-Ricardian equivalence)
\item data difficult to measure
\item slow policies
\item lobbies
\end{enumerate}
}

\Section{Chapter 17. Permanent Changes in Monetary and Fiscal Policy}

\frame{
\frametitle{Permanent Changes in Monetary Policy (1)}
Chapter 14:\\
An increase in a country's money supply:
\begin{itemize}
\item $R \Downarrow$
\item depreciation of the domestic currency
\end{itemize}
}


\frame[plain]{
\frametitle{Fig. 17-14: Short-Run Effects of a Permanent Increase in the Money Supply}
\begin{figure}
	\centering
		\includegraphics[width=0.90\textwidth]{sedici.pdf}
	\label{fig:11}
\end{figure}
}


\frame[plain]{
\frametitle{Permanent Changes in Monetary Policy (2)}
With $N>\bar{N} \Rightarrow w \Uparrow \Rightarrow P\Uparrow$
\begin{figure}
	\centering
		\includegraphics[width=0.90\textwidth]{diciasette.pdf}
	\label{fig:11}
\end{figure}
}



\frame{
\frametitle{Effects of Permanent Changes in Fiscal Policy}
If $G \Uparrow$ or $T \Downarrow$
The final effect is $D \Leftrightarrow$: an increase in government purchases completely crowds out net exports, due to the effect of the appreciated domestic currency
}


\frame[plain]{
\frametitle{Fig. 17-16: Effects of a Permanent Fiscal Expansion}
\begin{figure}
	\centering
		\includegraphics[width=0.90\textwidth]{diciotto.pdf}
	\label{fig:11}
\end{figure}
}

\Section{Chapter 17. Macroeconomic Policies and the Current Account (XX curve)}

\frame{
\frametitle{Macroeconomic Policies and the Current Account}
\begin{itemize}
\item XX curve: combinations of output and exchange rates at which the current account is at its desired level. 
\item if $Y \Uparrow \Rightarrow CA \Downarrow$
\item XX curve slopes upward but is flatter than the DD curve
\end{itemize}
}

\frame[plain]{
\frametitle{Fig. 17-17: How Macroeconomic Policies Affect the Current Account}
\begin{figure}
	\centering
		\includegraphics[width=0.90\textwidth]{diciannove.pdf}
	\label{fig:11}
\end{figure}
}

\frame[plain]{
\frametitle{Fig. 17-17: How Macroeconomic Policies Affect the Current Account}
\begin{figure}
	\centering
		\includegraphics[width=0.90\textwidth]{venti.pdf}
	\label{fig:11}
\end{figure}
}


\frame[plain]{
\frametitle{Fig. 17-18: The J-Curve}
\begin{figure}
	\centering
		\includegraphics[width=0.90\textwidth]{ventidue.pdf}
	\label{fig:11}
\end{figure}
}

\end{document}
