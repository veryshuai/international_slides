\documentclass[ignorenonframetext,]{beamer}

%Set up notes
\setbeamertemplate{note page}[plain]
\setbeameroption{hide notes}

\usepackage{amssymb,amsmath}
\usepackage{ifxetex,ifluatex}
\usepackage{fixltx2e} % provides \textsubscript
\usepackage{tikz}
\usepackage{tikz-qtree}
\usepackage{pdfpages} %include beamer pdf slides from other presentations
\usepackage{color}
\ifxetex
  \usepackage{fontspec,xltxtra,xunicode}
  \defaultfontfeatures{Mapping=tex-text,Scale=MatchLowercase}
\else
  \ifluatex
    \usepackage{fontspec}
    \defaultfontfeatures{Mapping=tex-text,Scale=MatchLowercase}
  \else
    \usepackage[utf8]{inputenc}
  \fi
\fi

% Comment these out if you don't want a slide with just the
% part/section/subsection/subsubsection title:
\AtBeginPart{
  \let\insertpartnumber\relax
  \let\partname\relax
  \frame{\partpage}
}
\AtBeginSection{
  \let\insertsectionnumber\relax
  \let\sectionname\relax
  \frame{\sectionpage}
}
\AtBeginSubsection{
  \let\insertsubsectionnumber\relax
  \let\subsectionname\relax
  \frame{\subsectionpage}
}

\setlength{\parindent}{0pt}
\setlength{\parskip}{6pt plus 2pt minus 1pt}
\setlength{\emergencystretch}{3em}  % prevent overfull lines
\setcounter{secnumdepth}{0}

\title{The Trade Workhorse: The Heckscher-Ohlin Model}
\author{Instructor: David Jinkins}
\date{Date: Sept. 11, 2014}

\begin{document}

\frame{\titlepage}

\begin{frame}
\begin{itemize}
\itemsep1pt\parskip0pt\parsep0pt
\item
  Last time

  \begin{itemize}
  \itemsep1pt\parskip0pt\parsep0pt
  \item The specific factors model:
      \begin{itemize}
          \item Analyze trade and the income distribution
          \item A short-run model: some factors can only be used to produce a single good
          \item Trade can hurt owners of some factors
          \item As trade ``grows the pie'', the right taxes can improve welfare for all
      \end{itemize}
      \item Political economy:
      \begin{itemize}
          \item Trade often unpopular as losses from trade are concentrated in an industry, while benefits are diffused
      \end{itemize}
  \item Migration:
      \begin{itemize}
          \item In the simplest model, migration hurts high wage labor
          \item Again, the pie grows, so that taxes can compensate
      \end{itemize}
  \end{itemize}
\end{itemize}

\end{frame}

\begin{frame}
\begin{itemize}
\itemsep1pt\parskip0pt\parsep0pt
\item
  Today: The Heckscher-Ohlin Model
  \begin{itemize}
        \item Introduction
        \item The point
        \item The model
        \begin{itemize}
            \item Setup and assumptions
            \item Production possibilities
            \item Production and prices
            \item Autarchy equilibrium
            \item Trade equilibrium
        \end{itemize}
        \item The big four theorems
        \item The problems 
        \begin{itemize}
            \item Assumptions
            \item Evidence
        \end{itemize}
        \item Moving forward
    \end{itemize}
\end{itemize}

\end{frame}

\begin{frame}
\begin{itemize}
    \item Quick note on this chapter of the textbook.
\end{itemize}

\end{frame}
\begin{frame}{Heckscher-Ohlin: Rise of a theory}

    \begin{itemize}
        \item From a book by Ohlin published in 1933 
        \item Original point $\rightarrow$ trade and long-run income distribution
        \item Attractive model
        \begin{itemize}
            \item Elegant: Can be analyzed graphically
            \item Enough complexity to tackle many trade issues
            \begin{itemize}
                \item Effect of trade liberalization
                \item Effect of tariffs
                \item Effect of technological change on trade patterns
                \item Effect of technological change on income distributions 
            \end{itemize}
            \item Clear, testable predictions
            \item Developed and extended by famous economists
            \begin{itemize}
                \item Ohlin (Nobel 1979)
                \item Samuelson (Nobel 1970)
            \end{itemize}
        \end{itemize}
    \end{itemize}

\end{frame}

\begin{frame}{Heckscher-Ohlin: Downfall}

    \begin{itemize}
        \item Some quotes: 
            \begin{itemize}
                \item \textcolor{blue}{\dots \emph{the Heckscher-Ohlin model is hopelessly inadequate as an explanation for historical and modern trade patterns, unless we allow for technological differences across countries.} }

-Robert Feenstra, Distinguished Professor of Economics at University of California, Davis, 2004
                \item \textcolor{blue}{\emph{It is time to declare Stolper-Samuelson [an important result of the HO model] dead. Stolper-Samuelson says that trade liberalization will raise the real income of the abundant (unskilled) labor in poor countries.  Stolper-Samuelson, qua theorem, is not wrong, of course. But if we use it, as we so often have, as if it provides a reliable answer to this question of real human significance, then it is worse than wrong - it is dangerous.} }

-Donald Davis and Prachi Mishra, 2007
            \end{itemize}
            
    \end{itemize}

\end{frame}

\begin{frame}{Heckscher-Ohlin: Why the hate?}

    \begin{itemize}
        \item Heckscher-Ohlin is a scientific theory: testable predictions
        \item Predictions have often not been backed up by data
        \item Assumptions also seem unusual in the modern world
    \end{itemize}

\end{frame}

\begin{frame}{The point}

    \begin{itemize}
        \item Hecksher-Ohlin about long-run effects of trade
        \item Embodied in the idea that all factors are costlessly mobile
    \end{itemize}

\end{frame}

\begin{frame}{The point}

    \begin{itemize}
        \item Famous implications are the four theorems (paraphrase):
        \begin{enumerate}
            \item \emph{Heckscher-Ohlin}: \textcolor{blue}{Countries with more of a resource will export goods for which that resource is more useful in production} 

\textcolor{red}{ex}: China exports labor intensive manufactured goods
            \item \emph{Rybczynski}: \textcolor{blue}{If a country gets more of a resource, it will more than proportionally increase its production of goods for which that resource is more useful in production}

\textcolor{red}{ex}: if Denmark got more labor, it would increase its production of textiles
            \item \emph{Stolper-Samuelson}: \textcolor{blue}{A change in the price of a final good for which a particular resource is more useful in production will increase the payments to that resource}

\textcolor{red}{ex}: if the price of textiles goes up, Chinese workers get higher wages
            \item \emph{Factor-price equalization}: \textcolor{blue}{Trade should cause factor prices to converge}

\textcolor{red}{ex}: Danish and Chinese workers should be paid the same wages
        \end{enumerate}
    \end{itemize}

\end{frame}

\begin{frame}{Model - Plan of attack}

        \begin{itemize}
            \item Setup and assumptions
            \item Production possibilities
            \item Production and prices
            \item Autarchy equilibrium
            \item Trade equilibrium
        \end{itemize}

\end{frame}

\begin{frame}{Setup}

        \begin{itemize}
            \item Two countries: Home (H) and Foreign (G)
            \item Two goods: Clothes (C) and Food (F)
            \item Two factors: Labor (L) and Capital (K)
            \item Sometimes called the 2 x 2 x 2 model
        \end{itemize}

\end{frame}

\begin{frame}{Setup}

        \begin{itemize}
            \item Clothes and Food can be produced with combinations of Labor and Capital
            \item Both countries have the same production technology 
            \item Production technologies are Constant Returns to Scale (double both inputs, double output)
            \item Each country has a different endowment of Labor and Capital 
            \item Labor and Capital cannot move between countries, even in the long run
            \begin{itemize}
                \item No shipping of machines allowed
                \item No moving abroad
            \end{itemize}
            \item As before, competitive firms with zero profit
        \end{itemize}

\end{frame}

\begin{frame}{Production Possiblities Frontier}

    \begin{itemize}
        \item Recall:   Slope = take a bit of Labor from Food production and put it into Clothes production
        \item Now also: Slope = take a bit of Capital from Food production and put it into Clothes production
        \item In notation: Slope = $-\frac{MPL_F}{MPL_C} = -\frac{MPK_F}{MPK_C}$ (why must ratios equal?)
        \item Why is the frontier concave?
    \end{itemize}
    \includegraphics[scale=0.15]{ppf.png}

\end{frame}

\begin{frame}{Input Possiblities}

    \begin{itemize}
        \item Fix the amount of, say, Food we want to produce 
        \item Slope = if I produce a bit more Food with Labor, how much Capital is freed up? 
        \item In notation: Slope = $-\frac{\frac{1}{MPK_F}}{\frac{1}{MPL_F}} = -\frac{MPL_F}{MPK_F}$ 
        \item Why is the curve convex?
    \end{itemize}
    \includegraphics[scale=0.15]{input_poss.png}

\end{frame}

\begin{frame}{Pause}

    \begin{itemize}
        \item We have described the environment
        \item We have described what it is possible to produce
        \item Now what will be produced in equilibrium?
        \item With what mix of factors?
    \end{itemize}

\end{frame}

\begin{frame}{Output price ratio and production}

    \begin{itemize}
        \item Suppose we have the price ratio $\frac{P_C}{P_F}$
        \item Then $\frac{P_C}{P_F} = \frac{MPL_F}{MPL_C} = \frac{MPK_F}{MPK_C}$ (why?)
        \item Production given by the point on the PPF tangent to $-\frac{P_C}{P_F}$
    \end{itemize}
    \includegraphics[scale=0.15]{ppf_equil.png}

\end{frame}

\begin{frame}{Input price ratio and production}

    \begin{itemize}
        \item Suppose we have the price ratio $\frac{w}{r}$
        \item Then $\frac{w}{r} = \frac{MPL_F}{MPK_F} = \frac{MPL_C}{MPK_C}$ (why?)
        \item Input bundle given by the point on the input possibilities curve tangent to $-\frac{w}{r}$
    \end{itemize}
    \includegraphics[scale=0.15]{input_cost.png}

\end{frame}

\begin{frame}{Input price change and production}

    \includegraphics[scale=0.25]{factor_price_change.png}

\end{frame}

\end{document}




